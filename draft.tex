
\documentclass[a4paper,abstracton]{scrartcl}
\pdfminorversion=5
%%%%%%%%%%%%%%%%%%%%%%%%%%%%%%%%%%%%%%%%%%%%%%%%%%%%%%%%%%%%%%%%%%%%%%%%%%%%%%%%%%%%%%%%%%%%%%%%%%%%%%%%%%%%%%%%%%%%%%%%%%%%%%%%%%%%%%%%%%%%%%%%%%%%%%%%%%%%%%%%%%%%%%%%%%%%%%%%%%%%%%%%%%%%%%%%%%%%%%%%%%%%%%%%%%%%%%%%%%%%%%%%%%%%%%%%%%%%%%%%%%%%%%%%%%%%
\usepackage{amsfonts}
\usepackage{amssymb}
\usepackage{amsmath}
\usepackage{amsthm}


\usepackage{dsfont} %For indicatior function \mathds{1}

\usepackage{accents}
\newcommand{\ubar}[1]{\underaccent{\bar}{#1}}

\usepackage{booktabs}
% \usepackage{algorithm}
% \usepackage{algorithmic}

\usepackage[ruled,vlined,linesnumbered]{algorithm2e}


\usepackage{graphicx}
\usepackage{color}

% \usepackage{subcaption}
\usepackage{subfigure}

\usepackage{a4wide}

%\usepackage{setspace}

%\newtheorem{conclusion}[theorem]{Conclusion}
%\newtheorem{condition}[theorem]{Condition}
% \numberwithin{equation}{section} 
% \numberwithin{table}{section}


\newtheorem{theorem}{Theorem}
\newtheorem{observation}[theorem]{Observation}
%  \newtheorem{proof}{Proof}[section]
\newtheorem{lemma}[theorem]{Lemma}
\newtheorem{prop}[theorem]{Proposition}
\newtheorem{remark}[theorem]{Remark}
\newtheorem{cor}[theorem]{Corollary}
\newtheorem{definition}[theorem]{Definition}
\newtheorem{example}[theorem]{Example}
% \numberwithin{observation}{section}


\usepackage{authblk}
\renewcommand\Affilfont{\small}

\allowdisplaybreaks

\DeclareMathOperator*{\argmax}{arg\,max}


\begin{document}




%%%%%%%%%%%%%%%%%%%%%%%%%%%%%%%%%%%%%%%%%%%%%%%%%%%%%%%%%%%%%%%%%
%%%%%%  Macros
%%%%%%%%%%%%%%%%%%%%%%%%%%%%%%%%%%%%%%%%%%%%%%%%%%%%%%%%%%%%%%%%%


%%%%%%%%%%%%%%%%%%%%%%%%%%%%%%%%%%%%%%%%%%%%%%%%%%%%%%%%%%%%%%%%%
\title{Distributionally Robust Optimization}

\author{Marc Goerigk}
\author{Michael Hartisch}

\affil{Network and Data Science Management, University of Siegen,\newline Unteres Schlo{\ss} 3, 57072 Siegen, Germany}	

    
\date{}

\maketitle



\begin{abstract}

\end{abstract}

\noindent\textbf{Keywords:} 

\section{First Example}
\subsection{First Try; Going along \cite{wang2019solution}}
We consider the selection problem where $p$ out of $n$ items have to be selected, while trying to minimize the resulting costs. The cost coefficients are uncertain and in particular random parameters.
 Following  \cite{wang2019solution} we assume that the distribution of the cost coefficients $\pmb{\xi}\in \mathbb{R}^n$ has a finite support, i.e. there are only $\vert\Omega\vert=N$ scenarios. 
\begin{align*}
\textnormal{(SEL)} \qquad\min \ & \alpha\\
\textnormal{s.t.}\ & \sum_{i \in [m]} x_i =p \\
&\mathbb{P}\left\lbrace\sum_{i \in [n]}\xi_i x_i \leq \alpha\right\rbrace \geq 1-\epsilon\\
& \pmb{x} \in \{0,1\}^n
\end{align*}

\begin{align*}
\textnormal{(DR-SEL)} \qquad\min \ & \alpha\\
\textnormal{s.t.}\ & \sum_{i \in [m]} x_i =p \\
&\inf_{\mathbb{P}\in \mathcal{P}}\mathbb{P}\left\lbrace\sum_{i \in [n]}\xi_i x_i \leq \alpha\right\rbrace \geq 1-\epsilon\\
& \pmb{x} \in \{0,1\}^n
\end{align*}

For now, $\alpha$ is a fixed value. In this case the above problem reduces to a feasibility problem and can be dealt with the techniques used in \cite{wang2019solution}. The binary reformulation of (SEL) is 

\begin{align*}
\textnormal{(SEL(Bin))} \qquad\min \ & \alpha\\
\textnormal{s.t.}\ & \sum_{i \in [m]} x_i =p \\
&\sum_{i \in [n]} \xi_i^\omega x_i + (M^\omega - \alpha) z_\omega \leq M^\omega && \forall\omega \in \Omega\\
&\sum_{\omega \in \Omega} p_\omega z_\omega \geq 1-\epsilon\\
& \pmb{x} \in \{0,1\}^n
\end{align*}
Note that in case of $\alpha$ being a variable the second constraint has to be linearized. In \cite{wang2019solution} upper bounds for $M^\omega$ are derived which can be done here similarly, in case of constant $\alpha$. The distributionally robust problem can be reformulated similarly:

\begin{align*}
\textnormal{(DR-SEL(Bin))} \qquad\min \ & \alpha\\
\textnormal{s.t.}\ & \sum_{i \in [m]} x_i =p \\
&\sum_{i \in [n]} \xi_i^\omega x_i + (M^\omega - \alpha) z_\omega \leq M^\omega && \forall\omega \in \Omega\\
&\inf_{\mathbb{P}\in \mathcal{P}} \sum_{\omega \in \Omega} p_\omega z_\omega \geq 1-\epsilon\\
& \pmb{x} \in \{0,1\}^n
\end{align*}
The above is semi-infinite due to the potential infinite ambiguity set $\mathcal{P}$. In case of Wasserstein ambiguity, due to its polyhedral structure, the respective constraint can be represented by its extreme points making the formulation finite.

\textcolor{red}{Potentielle Forschungsfrage: Kann man auch gute Bounds für $M^\omega$ bauen, wenn das $\alpha$ nicht fixiert ist? Dann anwenden auf allgemeine kombinatorische Probleme, Ich konnte bisher noch nicht überblicken ob ein variables $\alpha$ Probleme bereitet.}

For the case of Wasserstein ambiguity the dual reformulation was built in \cite{wang2019solution}.  They claim that they were not able to solve the dual formulation for none of their test cases within 10 hours.

\textcolor{red}{Potentielle Forschungsfragen: Kann man Laufzeit-Argumente für die gefundene dualen Formulierungen finden? Wie sieht die duale Reformulierung für andere Ambiguity sets aus? Lassen sich somit Komplexitätsaussagen für allgemeine kombinatorische Probleme treffen? Hier sollte vermutlich $\alpha$ immer auch variabel sein.}



\subsection{Selection problem as portfolio optimization}
The selection problem also can be interpreted as a portfolio optimization problem, where out of $n$ assets exactly $p$ have to be selected with similar weights, i.e. each of the $p$ assets will be added to the portfolio in the same proportion. For portfolio optimization a vast number of research articles can be found that deal with distributionally robust problems on risk measures. In the following we again assume a fixed set of scenarios $\Omega$ where their probability distribution lies within some ambiguity set $\mathcal{P}$. Additionally to maximizing the worst case payoff, a risk measure $\rho_p$ should be less than or equal to some fixed value $z$.

\begin{align*}
\textnormal{(DR-SEL-Risk)} \qquad\max \ & \alpha\\
\textnormal{s.t.}\ & \sum_{i \in [m]} x_i =p \\
&\sum_{i \in [n]}\xi^\omega_i x_i \geq \alpha && \forall \omega \in \Omega \\
&\rho_p(\pmb{x})\leq z && \forall p\in \mathcal{P}\\
& \pmb{x} \in \{0,1\}^n
\end{align*}
In the above problem, for a selection of $p$ assets the worst case payoff is maximized, while the selected risk measure is kept within a predefined limit $z$. 
For example, if the risk measure is the variance the constraint might look like this:

$$\sqrt{\sum_{\omega \in \Omega} p_\omega\left(\pmb{\xi^\omega}^\top \pmb{x}-\mathbb{E}_p[\pmb{\xi}^\top \pmb{x}]\right)^2} \leq z \qquad \forall p \in \mathcal{P}$$
Of course, instead of maximizing the worst case outcome, one could maximize the average outcome by using the following constraint:
$$\sum_{\omega \in \Omega}p_\omega\sum_{i \in [n]}\xi^\omega_i x_i \geq \alpha \qquad \forall p \in \mathcal{P}$$
Depending on the used risk measure and of course depending on the used ambiguity set, this problem can be reformulated to a tractable form \cite{postek2014tractable}.

\textcolor{red}{Bei den Risk-Measures sehe ich keine gute Einstiegsmöglichkeit, insbesondere da es schon sehr viel zu geben scheint.}

\subsection{Outlook on explorable distributionally robust problems}
What can we actually explore? This depends on the used ambiguity set and whether $\Omega$ is finite and predefined. For now we assume a finite, given scenario set $\Omega$. We might explore:
\begin{itemize}
\item The exact probability of at most $\Gamma$ scenarios. (budgeted explorability). Differ between handpicked and not handpicked scenarios?
\item The true expected value of $\pmb{\xi}^\top \pmb{x}$
\item the scenario with the largest/smallest probability
\item The true variance 
\end{itemize}
The explorable Setting:
\begin{itemize}
\item[1.] Select a query regarding the ambiguity set
\item[2.] The opponent answers the query
\item[3.] Select a solution
\item[4.] Opponent selects distribution from the ambiguity set that adheres to the answer of the query
\item[5.] Evaluate Solution
\end{itemize}

\bibliographystyle{alpha}
\bibliography{references}
\end{document}